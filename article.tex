\documentclass[lettersize,journal,english]{IEEEtran}
\usepackage[T1]{fontenc}
\usepackage[utf8]{inputenc}
\usepackage{amsmath,amsfonts,amssymb}
\usepackage{algorithmic}
\usepackage{algorithm}
\usepackage{array}
\usepackage[caption=false,font=normalsize,labelfont=sf,textfont=sf]{subfig}
\usepackage{textcomp}
\usepackage{stfloats}
\usepackage{url}
\usepackage{verbatim}
\usepackage{graphicx}
\usepackage{cite}
\usepackage{booktabs}
\usepackage{units}
\usepackage[unicode=true,pdfusetitle,
 bookmarks=true,bookmarksnumbered=false,bookmarksopen=false,
 breaklinks=false,pdfborder={0 0 0},pdfborderstyle={},backref=false,colorlinks=true]{hyperref}
\hypersetup{
 urlcolor=blue, linkcolor=black}
\usepackage[acronym]{glossaries}
\hyphenation{op-tical net-works semi-conduc-tor IEEE-Xplore}

\global\long\def\BDsites{\textsf{BD\_sites}}

\title{Adaptative neigbouring methods applied to telephonic base stations}
\author{\IEEEauthorblockN{Paul MÉHAUD}\\
\IEEEauthorblockA{\textit{Intern at CTU in Prague} \\
\textit{INSA Rouen Normandie}\\
paul.mehaud@insa-rouen.fr}\\
\and
\IEEEauthorblockN{Brendan SÉVELLEC}\\
\IEEEauthorblockA{\textit{Intern at CTU in Prague} \\
\textit{INSA Rouen Normandie}\\
brendan.sevellec@insa-rouen.fr}}

\makeglossaries

\setglossarypreamble{All definitions are from \href{https://www.dictionary.com/}{Dictionnary.com}}

\newglossaryentry{base station}
{
    name=base station,
    description={A fixed transmitter that forms part of an otherwise mobile radio network}
}

\newglossaryentry{convHull}
{
    name=convex hull,
    description={The smallest convex set containing a given set}
}



\newacronym{arcep}{ARCEP}{French Regulatory Authority for Electronic Communications and Posts}
\newacronym{anfr}{ANFR}{French National Frequency Agency}
\newacronym{bs}{BS}{\Gls{base station}}
\newacronym{dbscan}{DBScan}{Density-Based Spatial Clustering of Applications with Noise}
\newacronym{hdbscan}{HDBScan}{Hierarchical Density-Based Spatial Clustering of Applications with Noise}

\usepackage[english,french]{babel}

\begin{document}
\selectlanguage{english}
\maketitle

\begin{abstract}
  The field of telecommunication represents a gigantic mine of information, the smartphone penetration rate being 69\%
  in 2023 in the world, and 97\% in France. Therefore, the collected datas represent a big opportunity for telecommunication
  companies to predict their customers behaviour for internal or marketing purposes. Thus, it is primordial for these companies
  to be able to determinate if a user is moving. For that, it is necessary to understand the neighbouring relationships between the 
  telephonic base stations. This article aims at comparing different method, ellaborated in order to adaptively detect these
  relationship between telephonic base stations given their geographical positions.
\end{abstract}

\section{Introduction}
    \IEEEPARstart{T}{his} article is the continuation of the work done by Delphine Paquiry in this paper \cite{art_del_paq}.
    It aims at discovering the neighbouring relationships between base stations from their geographical positions, using a 
    measure of the density of base station.

    For that, it is necessary to describe and analyse the databases that have been used during the research, then to present 
    the different approaches that always consist of these steps: establish a first neighbouring graph, compute the base station 
    densities and filter this graph with different criteria using this information. The final step is to present the results
    of the different approaches and conclude.
\section{Related works}

\section{Databases}
\noindent Several databases have been used in order to test the different aspects of the methods. However, one of
them is directly necessary to the method itself.

\subsection{Main database}
This database \cite{main_database} contains all the information needed to the application of the method. It is from
\acrfull{arcep}, which is an independent French administrative authority responsible for regulating electronic and postal communications and press distribution.

The fields that are being used here are described in Table~\ref{table:data_columns}. [expliquer qu'il y x colonnes mais qu'on utilise que elles]
\begin{table}[!b]
    \centering
    \caption{Description of the main database columns used in the experiments}
    \label{table:data_columns}
    \begin{tabular}{ll}
        \toprule
        \textbf{Column} & \textbf{Description} \\
        \cmidrule(lr){1-2}
        \textsl{id\_station\_anfr} & \acrshort{anfr} \acrshort{bs} ID \\ 
        \textsl{nom\_op} & Name of the provider \\
        \textsl{x, y, latitude, longitude} & Base station coordinates \\ 
        \textsl{nom\_reg, nom\_dep, nom\_com} & Additional location information \\  
        \textsl{site\_2g, 3g, 4g, 5g} & Technology used by the \acrshort{bs} \\ 
        \bottomrule
    \end{tabular}
\end{table}

This database is updated every quarter. For instance, the own we used was published on 28\textsuperscript{th} March 2024 \cite{main_database_hist}. [dire depuis quand la base existe]

On the site there is a chat where we can ask questions about the databases \cite{main_database_chat}.

[parler des technologies, rajouter des graphes]

\begin{table}[!t]
    \centering
    \caption{Number of \acrfull{bs} equipped, at least, with $x$G, by provider}
    \label{table:techno_numbers}
    \begin{tabular}{cccccc}
        \toprule
        \textbf{Technology} & \textbf{Bouygues} & \textbf{Free} & \textbf{Orange} & \textbf{SFR} & \textbf{Total} \\
        \cmidrule(lr){1-6}
        \textbf{2G} & 21710 & 0 & 18835 & 20093 & 60638 \\
        \textbf{3G} & 26294 & 25934 & 29748 & 25777 & 107753 \\
        \textbf{4G} & 26231 & 25847 & 30440 & 25881 & 108399 \\
        \textbf{5G} & 11271 & 18607 & 8794 & 10968 & 49640 \\
        \textbf{Total} & 26331 & 25949 & 30540 & 26018 & 108838 \\
        \bottomrule
    \end{tabular}
\end{table}

\begin{figure}[!b]
    \centering
    \boxed{\includegraphics[width=2.5in]{images/data_analysis/with_techno.png}}
    \caption{Breakdown, by operator, of the number of sites containing a given technology}
    \label{fig:data_technos}
\end{figure}

\subsection{Filtering the database}
As seen before this database is really huge, as a matter of fact we will only work on a portion of it.

Firstly, only one provider will be kept because a \acrshort{bs} from provider $x$ can't be considered as a neighbour of \acrshort{bs} from provider $y$.
Consequently, we have chosen \emph{Orange} because it is the one that has the largest number of \acrshort{bs}.

Secondly, for the same reasons as before, only one technology is kept. Thus, we kept all \acrshort{bs} using, at least, 4G.

Finally, in order to simplify viewing, we will apply our neighbouring only on the Normandie region just because this is where we live.
\section{Reminder on existing methods}
\subsection{Finding potential neighbours}
When we are looking for neighbours, we need, at first, a list of potential neighbours for each point.
For that, we will construct a neigbouring graph $G = (P, U)$ where $P$ is the set of all points and each edge in $U$ represents a neighbouring relation.

\subsubsection{$k$-NN graph}
The most intuitive method is to connect each point to its $k$ nearest neighbours, where $k \in \mathbb{N}^*$ is a parameter to be fixed. 

However, this method can be criticised, because it always finds $k$ neighbours for each point, no matter the reality of the 
data.

\subsubsection{Delaunay triangulation}
A Delaunay triangulation \cite{art_delaunay} of a set of points in the plane subdivides their \gls{convHull} into triangles whose circumcircles 
do not contain any of the points.

A Voronoi diagram is a tessellation pattern in which $n$ points scattered on a plane subdivides in 
exactly $n$ cells enclosing a portion of the plane that is closest to each point. 

It is a well-known method when trying to find neighbours \cite{delaunay_neighbor}
since a Delaunay triangulation is the pendant of a Voronoi diagram, cf. Figure~\ref{fig:del_tri}, i.e. two points are connected in the
Delaunay triangulation if their associated cells are touching each other in the Voronoi diagram.

\begin{figure}[!b]
    \centering
    \boxed{\includegraphics[width=1.25in]{images/illus_graphs/Delaunay_circumcircles_vectorial.png}}
    \boxed{\includegraphics[width=1.25in]{images/illus_graphs/Delaunay_Voronoi.png}}
    \caption{Example of a Delaunay triangulation (in black) and Voronoi diagram (in red)}
    \label{fig:del_tri}
\end{figure}

\subsubsection{Gabriel graph}
A Gabriel graph \cite{10.2307/2412323} is a subgraph of a Delaunay triangulation. Thus, if the Delaunay triangulation is given, it can be found in a linear time. 

Formally, it is the graph in which any two distinct points $p, q \in P$ are adjacent precisely when the closed disc having $pq$ as a diameter contains no other points.

Basically, the interset of this method, compared to the Delaunay triangulation, is that it is adding a first level of filtration to suppress some neighbouring connexions.

\subsection{Finding real neighbours}
For the purpose of our experiments, i.e. finding \acrshort{bs} neighbours, the previously presented methods are not sufficient to find real neighbours.
As a consequence, some neighbouring connexions needs to be suppressed \cite{patent_neighs}.
\subsubsection{Distance criterion\cite{patent_neighs}}
A \acrshort{bs} does not have an infinite coverage area.
As a consequence, each edge in $U$ that is longer than a certain value, called \emph{max\_distance}, is suppressed.

\subsubsection{Angle criterion}
Let $p$, $q$, $r$ be three \acrshort{bs} of $P$, where $q$ and $r$ are $p$'s potential neighbours. As said before, coverage areas of \acrshort{bs} are limited by physics, this also implies that $r$ can be \og hidden\fg{} from $p$ by $q$ because $q$ is between $p$ and $r$ [trouver qqch pour prouver ce que je dis].

Therefore, the connexion between $p$ and $r$ needs to be suppressed. How? The angle $\widehat{qrp}$ is calculated and if it is narrower than a certain value, called \emph{min\_angle}, the longest edge is suppressed.

\subsubsection{Quadrant criterion}
\begin{figure}[!b]
    \centering
    \boxed{\includegraphics[width=2.5in]{images/illus_crit/quadrant_elim.png}}
    \caption{Example of a Delaunay triangulation}
    \label{fig:crit_qua}
\end{figure}

As introduced in a previous article \cite{10201211} its goal is to prevent all potential neighbours to be in the same angle range. Therefore, it was mainly thought to be used with the 
$k$-NN potential neigbours method.

It consists in dividing the space around each base station into quadrants and keeping only a certain number of potential neighbours
in each of the quadrants, cf. Figure~\ref{fig:crit_qua}.
The number of quadrants has been fixed to six \cite{art_del_paq}.

\subsection{Density-based clustering methods}
\subsubsection{\acrshort{dbscan}}
\acrshort{dbscan} consists in detecting clusters based on the density of points, using two parameters:
\begin{itemize}
    \item \emph{$\varepsilon$}: A distance threshold under which two points are considered close enough to be considered neighbours;   
    \item \emph{$n_{\text{min}}$}: The shortest number of neigbours a points needs to have to be considered as a core point of a cluster.
\end{itemize}

The clusters will then \og grow\fg{} from the core points by attaching all points close from less than \emph{$\varepsilon$} than a point of the said 
cluster.

\subsubsection{\acrshort{hdbscan}}
\acrshort{hdbscan} \cite{10.1007/978-3-642-37456-2_14} is a variation of the \acrshort{dbscan} method. It consists
in selecting the best clusters after testing their persistence by making the $\varepsilon$ parameter vary. It's main particularity
is that it allows to detect clusters of different densities.

To each point who belongs to a cluster is associated a probability that symbolises the degree of certainty for this point to belong
to the said cluster.

\section{Contributions}
\noindent Each of the precedent methods gives us a list of potential neighbours for each \acrshort{bs}.
However, we know that some of this neighbouring connexions are wrong,
either because of the neighboring graph construction method or physical specificities linked to the base stations themselves.

Thus, we need to find methods to detetect and erase these bad connexions.

The main innovation of our work compared to this one \cite{art_del_paq} is to take in account the difference of \acrshort{bs}'s density.

\subsection{Measurement of \acrshort{bs} density}
Places with a high population also have a high density of base stations. In fact, each base station can only support a certain
load of users.

Therefore, we can assume that finding zones with a high density of base station is equivalent to finding the \acrshort{bs} 
situated in \og cities\fg{}.

Moreover, to simplify the problem, each \acrshort{bs} will be classified in one of this four categories:
\begin{itemize}
    \item \emph{city center};
    \item \emph{urban area};
    \item \emph{extra-urban area};
    \item \emph{countryside}.
\end{itemize}

\subsubsection{\acrshort{dbscan}}
The goal here is to separate points that seem to belong to a denser conglomerate of base station, from the others who are a bit more spaced
to cover the countryside.

That is called density based clustering. The main method to do that is \acrshort{dbscan}.

It consists in detecting clusters based on the density, using two parameters:

\begin{itemize}
    \item \emph{$\varepsilon$}: A distance threshold under which two points are considered close enough to be considered neighbours;   
    \item \emph{$n_{\text{min}}$}: The shortest number of neigbours a points needs to have to be considered as a core point of a cluster.
\end{itemize}

The clusters will then \og grow\fg{} from the core points by attaching all points close from less than \emph{$\varepsilon$} than a point of the said 
cluster.

Despite the efficiency of this method, it is not the best adapted to our problem because it's only giving us if a \acrshort{bs} is considered 
in a city or not, it doesn't split the points that are in a city into the four categories presented above.

\subsubsection{\acrshort{hdbscan}}
This method, introduced here \cite{10.1007/978-3-642-37456-2_14} is a variation of the \acrshort{dbscan} method. It consists
in selecting the best clusters after testing their persistence by making the $\varepsilon$ parameter vary. It's main particularity
is that it allows to detect clusters of different densities.

To each point who belong to a cluster is associated a probability that symbolises the degree of certainty for this point to belong
to the said cluster.

This probability can be used to separate the center of the city (where the probability is very high) from the rest of the city. However,
it cannot be precise enough to separate the clusters into the four categories.

\subsubsection{$3$-NN mean distance}
For our use case, we need a more specific method. Here is the explanation of how it works:

For each base station $i$, the mean distance to its three nearest neighbours is calculated (we will call this value $\gamma_i$, in 
$\unit{km}$). According to the value of $\gamma_i$, the base station can either be classified as part of one of the categroies of 
Table~\ref{table:crit_summary} (very low value for city center, slighlty higher value for urban area, \dots).

To make this method work properly, only \acrshort{bs} from one provider have to be kept into the calculation process.

Moreover, this values are working well on France but it is not proven that it can work in another country. If someone uses this methods, 
values could be modified [dire que c'est plutot qqch à faire dans le futur et pas que c'est pas vérifié].

[Rajouter les catégories]

\subsection{Filtering criteria}

\begin{table}[!b]
    \centering
    \caption{Summary of criteria values}
    \label{table:crit_summary}
    \begin{tabular}{lcc}
        \toprule
        \textbf{Description} & \textbf{\emph{max\_distance}} & \textbf{\emph{min\_angle}} \\
        \cmidrule(lr){1-3}
        city center & $\unit[2]{km}$ & $5^\circ$ \\
        urban area & $\unit[5]{km}$ & $10^\circ$ \\
        extra-urban area & $\unit[10]{km}$ & $15^\circ$ \\
        countryside & $\unit[15]{km}$ & $20^\circ$ \\
        \bottomrule
    \end{tabular}
\end{table}

\begin{figure}[!t]
    \centering
    \boxed{\includegraphics[width=2.5in]{images/illus_crit/points.png}}
    \caption{Example of a Delaunay triangulation}
    \label{fig:crit_pts}
\end{figure}

\begin{figure}[!b]
    \centering
    \boxed{\includegraphics[width=2.5in]{images/illus_crit/distance_elim.png}}
    \caption{Example of a Delaunay triangulation}
    \label{fig:crit_dis}
\end{figure}

\begin{figure}[!t]
    \centering
    \boxed{\includegraphics[width=2.5in]{images/illus_crit/angle_elim.png}}
    \caption{Example of a Delaunay triangulation}
    \label{fig:crit_ang}
\end{figure}

\begin{figure}[!t]
    \centering
    \boxed{\includegraphics[width=2.5in]{images/illus_crit/neighs.png}}
    \caption{Example of a Delaunay triangulation}
    \label{fig:crit_nei}
\end{figure}

\subsubsection{Distance criterion}
When looking at the results of Delaunay triangulation, we realise that some edges, mostly on the border of $P$ are too long.

Indeed, a \acrshort{bs} does not have an infinite coverage area. As a consequence, for each \acrshort{bs} the neighbouring connexions that are longer than a certain value, called \emph{max\_distance} (cf. Table~\ref{table:crit_summary}), are suppressed.

\subsubsection{Angle criterion}
Let $p$, $q$, $r$ be three \acrshort{bs} of $P$, where $q$ and $r$ are $p$'s potential neighbours. As said before, coverage areas of \acrshort{bs} are limited by physics, this also implies that $r$ can be \og hidden\fg{} from $p$ by $q$ because $q$ is between $p$ and $r$ [trouver qqch pour prouver ce que je dis].

Therefore, the connexion between $p$ and $r$ needs to be suppressed. How? The angle $\widehat{qrp}$ is calculated and if it is narrower than a certain value, called \emph{min\_angle} (cf. Table~\ref{table:crit_summary}), the longest edge is suppressed.

[Dire que c'est des angles adjacents]

\subsubsection{Quadrant criterion}
As introduced in a previous article \cite{10201211} its goal is to prevent all potential neighbours to be in the same angle range. Therefore, it was mainly thought to be used with the 
$k$-NN potential neigbours method.

It consists in dividing the space around each base station into quadrants and keeping only a certain number of potential neighbours
in each of the quadrants. The number of quadrants has been fixed to six \cite{art_del_paq}.

The criterion has been upgraded by adding the possibility to rotate the quadrants. The offset angle selected, which is between $0$ and 
$60$ degrees, is the one that maximises the sum of the distances of the potential neighbours to their closest quadrant delimitation.

In other words, it tends to select the angle with which the potential neighbors are as much in the middle of their respective quadrants
as possible.

The quadrant criterion is illustrated in Fig~\ref{fig:crit_qua}.

\section{Results}

\subsection{Combining criteria}
As a matter of optimisation, criteria are applied one after another. For example, the base graph is Delaunay triangulation and we apply distance then angle.
The principle is that angle criterion will be applied on the Delaunay graph filtered with the distance criterion.

As a consequence, results differ according to the order in which the criteria are applied.

The best is Delaunay with distance and then angle.

\section{Conclusion}

[Ouverture: prendre en compte les zazimuths]

\printglossary[type=\acronymtype]
\printglossary

\bibliographystyle{IEEEtran}
\bibliography{./biblio.bib}

\end{document}


